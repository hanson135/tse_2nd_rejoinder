% Copyright Javier Sánchez-Monedero.
% Please report bugs and suggestions to (jsanchezm at uco.es)
%
% This document is released under a Creative Commons Licence 
% CC-BY-SA (http://creativecommons.org/licenses/by-sa/3.0/) 
%
% BASIC INSTRUCTIONS: 
% 1. Load and set up proper language packages
% 2. Complete the paper data commands
% 3. Use commands \rcomment and \newtext as shown in the example

\documentclass[a4paper,twoside,11pt]{reviewresponse}

% 1. Load and set up proper language packages
%\usepackage[utf8x]{inputenc}
\usepackage[latin9]{inputenc}
\usepackage[T1]{fontenc}
\usepackage[english]{babel}
\usepackage{eurosym}
\usepackage{soul}

\usepackage[normalem]{ulem}

%henry.edison@inf.unibz.it,smorsgar@stud.ntnu.no, xiaofeng.wang@unibz.it, pekka.abrahamsson@jyu.fi
% 2. Complete the paper data
\newcommand{\myAuthors}{{Henry~Edison$^{\displaystyle a}$, ~Xiaofeng~Wang$^{\displaystyle b}$, ~Kieran~Conboy$^{\displaystyle a}$}}
\newcommand{\myAuthorsShort}{Henry~Edison et. al}
\newcommand{\myEmail}{}
\newcommand{\myTitle}{Response to Reviewers: \\Comparing Methods for Large-Scale Agile Software Development: A Systematic Literature Review}
\newcommand{\myShortTitle}{Response to Reviewers}
\newcommand{\myJournal}{Ref. No. TSE-2020-03-0124.R1 }
\newcommand{\myDept}{{$^{\displaystyle a}$Lero, NUI Galway, Ireland} \\%\\ \url{http://www.uco.es/ayrna/}}\\
{$^{\displaystyle b}$Free University of Bozen-Bolzano, Italy}
%{$^{\displaystyle c}$Faculty of Information Technology, P.O. Box 35, FI-40014, University of Jyv\"{a}skyl\"{a}, Finland }
}
%%%%%%%%%%%%%%%%%%%%%%%%%%%%%%%%%%%%%%%%%%%%%%%%%%%%%%%%%%%%%%%%%%%%%%%%%%


%\usepackage[linktoc=all]{hyperref}
\usepackage[linktoc=all,bookmarks,bookmarksopen=true,bookmarksnumbered=true]{hyperref}

\hypersetup{
pdfauthor = {\myAuthorsShort},
pdftitle = {\myTitle},
pdfsubject = {\myJournal\xspace},
colorlinks = true,
linkcolor=black!70!green,          % color of internal links
citecolor=black!70!green,        % color of links to bibliography
filecolor=magenta,      % color of file links
urlcolor=black!70!green           % color of external links
}

\begin{document}

\thispagestyle{plain}

\begin{center}
 {\LARGE\myTitle} \vspace{0.5cm} \\
 {\large\myJournal} \vspace{0.5cm} \\
 \today \vspace{0.5cm} \\
 \myAuthors \\
 \url{\myEmail} \vspace{1cm} \\
 \myDept
\end{center}

%\tableofcontents

%\begin{abstract}
% We thank the anonymous Reviewers for their efforts and feedback.
%\end{abstract}
\clearpage

\section{Associate Editor}
\rcomment{The paper looks relevant to TSE, and so far, it seems to provide a valuable contribution and should be published, given that the comments provided here are appropriately addressed. Our editorial board and expert reviewers determined that the paper seems to have different essential weaknesses to be seriously considered for publication. Among others, the primary concerns expressed were:
\begin{itemize}
	\item Mapping of challenges/SF to those in previous work
	\item Feedback on the authors' approach to the improvement suggestions and recommendations w.r.t. the original submission. Among others, the authors should take into account:
	\begin{itemize}
		\item Exclusion of work that is potentially relevant to this literature review revision of the factor related to failure
		\item Thorough feedback related to the new text added, which has been presented with accurate detail
	\end{itemize}
	\item Readability of the paper, in general
\end{itemize}
}
\textbf{Response:}
We thank you for these comments. In the current version, we have addressed the concerns as follows: 
\begin{itemize}
	\item \textit{Mapping of challenges/SF:}
	\item \textit{Feedback on the suggestion and recommendation:} We have improved all the texts pointed by the reviewer team. We hope now it has been clearer and more accurate.
	\item \textit{Readability of the paper:} Based on the comments from the reviewer team, we have improved
each section of the paper. The paper have been also proofread by an English native speaker.
\end{itemize}

\clearpage

\section{Reviewer 1}
\rcomment{The only response that I find a bit shallow is the one of Comment 3 (mapping the challenges/SF to the ones of previous works). I understand that the work addresses the application of specific large-scale agile methods whereas Dikert et al. [6] address large-scale agile transformations. Nevertheless, I think that there is potential to improve the paper. In Dikert et al. [6], specific methods (e.g., SAFe and LeSS) are only mentioned in the future research agenda. Dikert et al.[6] report challenges/SF without using specific methods. It would be beneficial to explicitly indicate which challenges/SF seem to be tackled by using specific methods, which ones remain, and which ones arise. For instance, (C-IC) Inter-team Coordination (from the paper) and Coordination challenges in multi-team environment (from Dikert et al. [6]) are related. It seems that despite of using specific large-scale agile methods that challenge remains. \\

The above would be helpful to understand the pros and cos of using specific large-scale methods in comparison to non large-scale methods.}
\textbf{Response:}
We thank you for the comment. 

\clearpage

\section{Reviewer 2}
\rcomment{I found that the revised paper sufficiently addressed the comments I raised. (1) I asked for an expanded Discussion to include implications for theory and practice. This has been added and greatly contributes to the improvement of the paper. (2) I raised two concerns about how the original manuscript cited and addressed previous literature reviews, including erroneous summaries and descriptions. These have been addressed with sufficient corrections and adjustments to the manuscript. \\

Thus, I feel that all my previous concerns have been sufficiently addressed and I appreciate the work done by the authors to address them.
}
\textbf{Response:}
We thank you for the valuable feedback to improve our paper and are glad that you are happy with the revision.

\clearpage

\section{Reviewer 3}
\rcomment{A. I am very much concerned that for some relevant papers that I indicated in my review, the authors state that they did not include these sources just because these sources did not have ``the right key words''. Well, I think this is not a good reason to exclude potentially relevant. We know that each journal and conference has a tacit culture of choosing keywords (of course there are accepted rules but adding key words may often go well beyond the explicit rules). Barbara Kitchenham suggests that we build our search strings as a result of experimentation by gradually expanding our strings in order to be able to hit all those papers that we know are relevant. \\

If you chose not to follow this approach to hitting the most relevant papers, I will be fine with your choice but please assure the following:
\begin{enumerate}
	\item You provide argumentation of why you decide to ignore certain papers
	\item Provide discussion on validity due to this. By leaving out some potentially relevant papers, you accept to inject some validity threats to the research process, and this calls for a discussion and thinking of some measures to mitigate the risk that the review will be incomplete due to missing work (missing out on potentially relevant empirical papers)
\end{enumerate}}
\textbf{Response:}
We thank you for the comment. As we described in Section 3.1, our search strings did not come from a vacuum, but were informed by existing literature reviews on large-scale agile and the VersionOne report. Moreover, we have defined our selection criteria in Table 1. We realised that research on large-scale agile development is a cross-disciplinary study, and so terms used to describe the phenomenon are inconsistent. Therefore, in Section 3.5.2 we acknowledged the threat of missing relevant papers. However, to minimise the threat of missing excluding relevant papers, our strategy was to retrieve as many peer reviewed papers as possible in both SE and IS digital libraries. We also performed a snowballing approach (backward and forward) as suggested by Kuhrmann et al (2017). In Section 3.5.3, we discussed our strategy to minimise the threat to consistency of selection primary studies and data extraction. To be included in the analysis, two reviewers had to agree on each individual paper.

%Our search string comprises of three groups which was separated by AND- clauses. The first group was to ensure that the methods used in large-scale development are agile methods. Second group was to capture studies on software engineering or information system development. Finally, the third group was to scope the search process on large-scale settings only. Therefore, to be included in the review process, a paper should report the use of agile methods for software development in large-scale setting. 

\rcomment{B. Please see in the Response to Reviewers, Comment 12: You state in the Response to Reviewers that you do not look at failure factors. However, in your letter you also state: ``Challenges are issues that may cause project delays, quality issues, or failure if not addressed [27], [28], [29].''  So, what is the relationship between challenge that you define right now and the term ``failure''?}
\textbf{Response:}
We thank you for the comment. We have updated the text as follows: \\
``Challenges are issues or obstacles that demand great consideration and need to be overcome [27]. When challenges are unmanaged, they may cause project delays, quality issues, or failure [28], [29].''

\rcomment{C.
Further on your comment 12: ``Most existing studies focus on either challenges or success factors but not both.'' Yes, I agree, but why this is essential? Would not we appreciate the collective body of knowledge, and not the fact that the challenges are researched by one team of authors, and the success factors - by another team? Yes, you juxtaposed these two, but in what way this is different compared to positioning two works (one on challenges and one on successes) next to each other? I think it is OK in any discipline to have parts of the whole researched by different teams. Sorry, but this on its own is not a contribution: -) if you chose to treat them both, this is a desktop decision and I agree with it as this is your research protocol.}
\textbf{Response:}
We thank you for the comment. As we discussed in Section 2, that from the 10 existing literature reviews on large-scale agile, only three reviews reported the challenges and success factors or large-scale agile development in general rather than method specific. Two reviews specifically looked at challenges and success factors of individual method such SAFe (Putta et al 2018) and SAFe and LeSS (Kalenda et al. 2018). Also, none of these reviews reported on custom-built methods. Our contribution instead is to identify and summarise all the findings on all commercial, custom-built and scaled agile methods reported in the literature. We have updated the text as follows: \\
``Most existing literature reviews on large-scale agile methods focus on either challenges or success factors but of one or two commercial methods (SAFe and LeSS) or large-scale agile transformation.''

\rcomment{D. See Comment 19: I see you acknowledge that there are many challenges that are also those that we know from 40 years ago and that are usual for large scale projects (waterfall). I think, in the discussion you should add a paragraph on this. Please mark those that are new and attributable to agile. And comment that others might not be attributable to agile but to the fact that the project is large scale. In fact, if a challenge exists just because the project is big, then we do not assume that agile large scale will make this challenge disappear. I think that many of the old challenges will stay, but it is possible that new and unexpected challenges would come up -- and this is where your paper adds new knowledge (it shows the collective list of new challenges that we did not know in large scale project from the time before agile).}
\textbf{Response:}
We thank you for this comment. We did discuss this point at length and gave it serious consideration. We believe we have addressed it using the text below (see the last paragraph in Section 6.4): \\

``When one considers the source of the challenges and success factors identified in this study, it is logical that some arise due to the large-scale nature of the development context. For example, inter-team co-ordination (C- IC) has been an issue associated with large-scale projects long before agile (e.g. [64], [65]). Similarly, some challenges and success factors are associated with agile development, even in small (non-large-scale) development environs (e.g. requirements engineering challenge (C-RE) has been identified as a challenge even in the context of 2-5 developers [66]). However, while a challenge or success factor may be primarily attributed to large-scale or agile, our analysis of both sets of literature suggests that all make an appearance in either set, even though it may appear less so in one that the other. Further, even where a challenge may appear in large-scale generally pre-agile (e.g. C-IC), the agile nature of work certainly exacerbates it. For example, the traditional pre-agile response to co-ordination of large teams would be to control from the top down by appointing a set of controllers responsible for ensuring procedures are adhered to using extensive standardised reporting over long periods of time [67], [68]. However, agile principles forbid top down control, extensive reporting and any intolerance of change. Therefore, while co-ordination is a long-standing challenge it is a much more complex one in an agile era, and one that requires different solutions to those of the past.''


\rcomment{1.
Please see Sect.1.1., 2nd column where you talk about the contributions: ``identify and present in one location various extensions to each of the original methods that have been proposed in the literature''. Sorry, but what ``location'' means in this sentence? I am confused... very sorry, you lost me here. What location do you mean? Please rephrase for clarity. \\

2. See in the same para, ``Until now, the original method was separated from any new recommended additions''. The article ``the'' hints to a method that has been introduced earlier. Which method do you refer to, here? This paragraph does not read well, please elaborate.}
\textbf{Response:}
We thank you for the comments. What we meant that in Table 6, based on the review findings, we presented various extensions to each of the prescribed large-scale agile methods (this is also to address the second comment about the article 'the') in terms of the principles, practices, tools, and metrics. We have updated the text as follows: \\
``In addition, we identify and present in one location (see Table 6) various extensions to each of the prescribed or commercial large-scale agile methods that have been proposed in the literature. Until now, these methods were separated from any new recommended additions that emerged through various pieces of empirical research.''


\rcomment{3. Please see: ``We decided to incorporate both because they are two inter-related and not completely overlapping facets of the application of large-scale agile methods.''  I do not understand what you mean with ``...not completely overlapping facets''. There is no published research that states that challenges and success factors are ``overlapping''. I therefore can not see what point you are trying to make here. Please rephrase or elaborate for clarity.}
\textbf{Response:}
We thank you for the comment. We have removed the text and updated the paragraph, as follows:\\
``Most existing literature reviews on large-scale agile methods focus on either challenges or success factors of one or two commercial methods (SAFe and LeSS) or large-scale agile transformation. We decided to incorporate both because they are two inter-related. However, tackling all challenges may not guarantee the successful application of these methods. Similarly, a company having all success factors in place can still encounter other challenges when applying large-scale methods. Inspecting both in one study enables us to produce the most comprehensive overview possible.''

\rcomment{4. In Section 4, before you start explaining what characteristics theories share, please add a linking sentence in the style ``For the purpose of this work, we adopt Gregor's understanding of ``theory''. The ref to Hannay et al can be dropped here and used later as an illustration). Adding such a sentence will make the text flow smoother.}
\textbf{Response:}
We thank you for the suggestion. We have fixed the text as follows:\\
``For the purpose of this work, we adopted Gregor's [56] understanding of theory, as follows: (i) generalisation -- an attempt to generalise knowledge of specific events or object into more abstract and universal, (ii) causality -- the relationship between cause and effect, and (iii) explanation and prediction -- understanding and predicting a phenomenon and guiding action.''

\rcomment{5.
See section 6.4. last sentence: ``While the timeliness and importance of these methods to practitioners is evidenced...'' Do you mean *evident*? Please rephrase for clarity.}
\textbf{Response:}
We thank you for the suggestion. Yes, we mean evident. We have fixed the issue.

\rcomment{7. The text needs polishing. there are long sentences that could be simplified (cut in 2) in order to improve readability. I recommend the authors give it a very careful polish, so that the paper reads smoothly. Here are some language mistakes that could be fixed (but there are more):
\begin{itemize}
	\item RQ1, needs a question mark at the end ``?''
	\item Scientific writing is about explicitness; please remove the word ``etc'' where it leaves open questions, see e.g. section 1.2. line 45. For example, earlier, we state DevOps; however, in line 45 DevOps is missing and is replaced by ``etc''. I suggest you drop ``etc'' and use ``DevOps'' instead
	\item See 5.4 RQ4 -- Success Factors of Applying Large-Scale Agile Methods; please correct ``...Factors **for the Application of** Large-scale Agile Methods''
	\item  See ref [57]. It should be ``use'' and not ``usee''
	\item Please check the references as well.
\end{itemize}
}
\textbf{Response:}
We thank you for the suggestion. We have fixed the issues.

\end{document}